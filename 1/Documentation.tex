\documentclass{article}
\usepackage{xcolor}
\usepackage{listings}
\lstset{basicstyle=\ttfamily,
  showstringspaces=false,
  commentstyle=\color{red},
  keywordstyle=\color{blue}
}
\begin{document}
\subsection{Approach}
    We Made an outline of the task we have to complete and broke it down into stages.
    In this case we prepared the following stages. 
    \begin{enumerate}
        \item Make a procedure (input) to take coordinates as input.
        Also display instructions to the user while asking for input. 
        \item Make a procedure (area) that takes 2 points and returns the area of the 
        trapeziod formed by those points. The area is calculated using single precision
        floating point registers. This procedure assumes that the input points are on the same side of the X axis.
        \item In the main procedure, using the procedure input, accept coordinates from the user.
        Call the procedure area using the coordinates received from the user. If two successive coordinates \( (x_1, y_1) \) and \( (x_2, y_2) \) are on different
        sides of the X-axis (which we determine by checking the sign of the product \( y_1y_2 \)), we break the area into 2 triangles, one lying on each side of the X-axis, and call the procedure
        area on each separately.
        \item An accumulator keeps track of the sum of all areas calculated, and after all the points have
        been processed, stores the area, which is to be displayed to the user. 
    \end{enumerate}
\subsection{Formulae used}
\begin{itemize}
    \item Area of trapeziod formed by \( (x_1, y_1) \) and \( (x_2, y_2) \)  when both points are above the X-axis is given by \[ Area = (y_2 - y_1)*(x_1 + x_2)/2 \].
    \item When both of the points are below X-axis then the negative of the above expression gives the absolute area.
    \item In the third case, we have that one point is above the X-axis and one is below. In this case, we introduce a third point, M on the X-axis where the line joining points A and B meet
    the X-axis. Then we can calculate the absolute area covered by A,B to be the sum of the areas of triangles formed by A,M and M,B. The coordinates of M are given by - \[ M = (x_1 -  y_1.(x_2 - x_1)/(y_2 - y_1),0)\]
\end{itemize}

\subsection{User Interface}
On running the program, we are first prompted to enter the number of points.
Then on each line we ask for the X or Y coordinates of the \(i\,th\) point.
After calculating the area is printed on the console with a message.

\subsection{Testing}
\subsubsection{Automated}
We wrote a python script that generates random coordinates, runs the asm file, and compares the
output with the expected area within a margin of error that can be specified while testing. We use spim command line
tool to automate the testing using the python script. To run the script, we enter the following commands once inside root project folder 
\begin{lstlisting}[language=bash]
    sudo apt install spim
    python3 -m pip install -r requirements.txt
    python3 tester.py -n 20 -m 100 -b 10 -e 0.0001 
\end{lstlisting}
This will run 20 random test cases with 100 coordinates of at max 10 bits each,
with an error tolerance of  0.0001 percent. \\
We tested with different values of error tolerance and bitsizes, and found that overflow occured at 16 bit inputs
and with 15 bit inputs, we could achieve an accuracy of about \(10^{-4}\,\%\). 

\subsubsection{Manual}
We also checked a few corner cases manually in QtSpim, which are listed below : 
\begin{enumerate}
    \item number of points 0 or 1
    \item all points on X-axis.
    \item All points on Y-axis.
    \item Duplicate points.
\end{enumerate}
In all these cases we got the expected output without overflows, that is 0.

\end{document}

